\documentclass{tufte-handout}
\usepackage{amsmath}
\usepackage{svg}
\usepackage{graphicx}
\setkeys{Gin}{width=\linewidth,totalheight=\textheight,keepaspectratio}
\graphicspath{{graphics/}}
\usepackage{booktabs}
\usepackage{units}
\usepackage{fancyvrb}
\fvset{fontsize=\normalsize}
\usepackage{multicol}
\usepackage{minted}
\usemintedstyle{bw}
\usepackage{FiraMono}
\usepackage[T1]{fontenc}
\usepackage{enumitem}
\definecolor{carnelian}{rgb}{0.7, 0.11, 0.11}
\usepackage{hyperref}
\hypersetup{colorlinks=true, linkcolor=carnelian, filecolor=carnelian, urlcolor=carnelian,}
\title{Transmogrify User Manual}
\author{Abhirag <hey@abhirag.com>}
\date{17 February 2022}
\begin{document}
\maketitle
\begin{abstract}
\textit{Hello and welcome to the Transmogrify user manual. 
Transmogrify is a tool to convert markdown sprinkled with some lisp to LaTeX. This user manual 
has also been generated using Transmogrify, so feel free to refer to its source for reference}
\end{abstract}\marginnote{
  \textsc{table of contents}
  \begin{enumerate}[nosep]
    \item \hyperref[sec:1]{Configuration}
    \item \hyperref[sec:2]{Lisp Extensions}
    \item \hyperref[sec:3]{Pikchr}
\end{enumerate}
}\section{Configuration}\label{sec:1}
Each document must start with a configuration code block:\begin{minted}[frame=leftline, framesep=10pt, fontsize=\footnotesize]{text}
```config
; Set the title of the document
(set-title "Transmogrify User Manual")

; Set the author of the document 
(set-author "Abhirag <hey@abhirag.com>")

; Set the date of the document
(set-date "17 February 2022")

; Set the width of Pikchr figures 
(set-pwidth 1000)

; Set the height of Pikchr figures 
(set-pheight 500)
```

\end{minted}
\section{Lisp Extensions}\label{sec:2}
We use an embedded lisp called \textit{fe} to extend the markdown syntax when necessary.
You write your document in plain markdown and use \textit{fe} code blocks for extra features
not supported by markdown such as:\subsection{Abstract}\begin{minted}[frame=leftline, framesep=10pt, fontsize=\footnotesize]{text}
```fe
(abstract "text")
```

\end{minted}
\subsection{Table of Contents}\begin{minted}[frame=leftline, framesep=10pt, fontsize=\footnotesize]{text}
```fe
(toc "Configuration" "Lisp Extensions")
```

\end{minted}
\subsection{Sidenote}
You can use inline \textit{fe} code to create a sidenote \sidenote[][0mm]{This is a sidenote}:\begin{minted}[frame=leftline, framesep=10pt, fontsize=\footnotesize]{text}
create a sidenote ``fe (sidenote "text" offset_in_mm)``

\end{minted}

Sometimes a sidenote may run over the top of other text or graphics in the margin space.
If this happens, you can adjust the vertical position of the sidenote by providing a numerical offset argument.\subsection{Marginnote}
If you'd like to place ancillary information in the margin without the sidenote mark (the superscript number),
you can use a marginnote \marginnote[0mm]{This is a margin note. Notice that there isn't a number preceding the note, and there is no number in the main text where this note was written}:\begin{minted}[frame=leftline, framesep=10pt, fontsize=\footnotesize]{text}
create a marginnote ``fe (marginnote "text" offset_in_mm)``

\end{minted}
\subsection{Text formatting in Lisp}
You can format text written in lisp using three functions \textit{italic}, \textit{bold} and \textit{concat}.
Here is a formatted marginnote as an example: \marginnote[0mm]{ \textit{This text is italic.} \textbf{This text is bold}}\begin{minted}[frame=leftline, framesep=10pt, fontsize=\footnotesize]{text}
``fe (marginnote 
       (concat 
         (italic "This text is italic.") 
         (bold "This text is bold")) 0)``

\end{minted}
\section{Pikchr}\label{sec:3}
Embedded \textit{Pikchr} code will be rendered to a diagram:\begin{minted}[frame=leftline, framesep=10pt, fontsize=\footnotesize]{text}
```pikchr
define ndblock {
  box wid boxwid/2 ht boxht/2
  down;  box same with .t at bottom of last box;   box same
}
boxht = .2; boxwid = .3; circlerad = .3; dx = 0.05
down; box; box; box; box ht 3*boxht "." "." "."
L: box; box; box invis wid 2*boxwid "hashtab:" with .e at 1st box .w
right
Start: box wid .5 with .sw at 1st box.ne + (.4,.2) "..."
N1: box wid .2 "n1";  D1: box wid .3 "d1"
N3: box wid .4 "n3";  D3: box wid .3 "d3"
box wid .4 "..."
N2: box wid .5 "n2";  D2: box wid .2 "d2"
arrow right from 2nd box
ndblock
spline -> right .2 from 3rd last box then to N1.sw + (dx,0)
spline -> right .3 from 2nd last box then to D1.sw + (dx,0)
arrow right from last box
ndblock
spline -> right .2 from 3rd last box to N2.sw-(dx,.2) to N2.sw+(dx,0)
spline -> right .3 from 2nd last box to D2.sw-(dx,.2) to D2.sw+(dx,0)
arrow right 2*linewid from L
ndblock
spline -> right .2 from 3rd last box to N3.sw + (dx,0)
spline -> right .3 from 2nd last box to D3.sw + (dx,0)
circlerad = .3
circle invis "ndblock"  at last box.e + (1.2,.2)
arrow dashed from last circle.w to 5/8<last circle.w,2nd last box> chop
box invis wid 2*boxwid "ndtable:" with .e at Start.w
```

\end{minted}
\begin{figure}
\fontfamily{lmss}\fontsize{8pt}{10pt}\selectfont  \includesvg[width=100mm,height=200mm,keepaspectratio]{3625387681806419601.svg}
\end{figure}

\end{document}